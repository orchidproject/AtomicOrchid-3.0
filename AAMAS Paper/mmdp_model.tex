\subsection{The Optimisation Problem}
\label{sec:model}
\noindent Previous agent-based models for team coordination in disaster response typically assume deterministic task executions and environments \cite{ramchurn:etal:2010,Scerri2005}. However, in order to evaluate agent-guided coordination in a real-world environment, it is important to consider uncertainties due to player behaviours and the environment (as discussed in the previous section). Given this, we propose novel representation for the task allocation problem in disaster response that does take into account such uncertainties. More specifically, we provide a novel representation for this problem using a Multi-Agent Markov Decision Process (MMDP) that captures the uncertainties of the radioactive cloud and the responders' behaviours ~\cite{boutilier1996planning}. We model the spreading of the radioactive cloud as a random process over the disaster space and allow the actions requested from the responders to  fail (because they refuse to go to a  task) or incur delays (because they are too slow) during the rescue process. Thus in the MMDP model, we represent  task executions as stochastic processes of state transitions while the uncertainties of the radioactive cloud and the responders' behaviours can be easily captured with transition probabilities.  More formally, the MMDP is
represented by tuple $\mathcal{M} = \langle I, S, \{A_i\}, P, R
\rangle$, where $I$ is the set of actors as defined in the previous
section,  $S$ is the state space, $A_i$ is a set of responder
$p_i$'s actions, $P$ is the transition function, and $R$ is the
reward function. We elaborate on each of these below.

More specifically, $S= S_r \times S_{p_1} \times \cdots \times
S_{p_n} \times S_{t_1} \times \cdots \times S_{t_m}$ where $S_r =
\{l_{(x,y)}| (x, y) \in G\}$ is the state variable of the
radioactive cloud that specifies the radioactive level $l_{(x,y)}\in[0,
100]$ at every point $(x, y)\in G$. $S_{p_i} = \langle h_i, (x_i,
y_i), t_j \rangle$ is the state variable for each responder $p_i$
that specifies his or her health level $h_i\in[0, 100]$, its present position
$(x_i, y_j)$, and the task $t_j$ they are carrying out. $S_{t_j}
= \langle st_j, (x_j, y_j) \rangle$ is then the state variable for
task $t_j$ to specify its status $st_j$ (picked up, dropped off, or
idle) and position $(x_j, y_j)$.

The three types of actions  (in set $A_i$) a responder can take
are: (i) {\em stay} in the current location $(x_i, y_i)$, (ii) {\em
move} to the 8 neighbouring locations, or (iii) {\em complete} a
task located in $(x_i, y_i)$. A joint action $\vec{a}=\langle a_1,
\cdots, a_n \rangle$ is a set of actions where $a_i\in A_i$, one
for each responder (a responder may just \emph{stay} at its current position if it has no targets to rescue). The transition function $P$ is defined in more detail as: $P= P_r
\times P_{p_1} \times P_{p_n} \times P_{t_1} \times P_{t_n}$ where:
\begin{itemize}
    \itemsep=-2pt
    \item $P_r(s'_r|s_r)$ is the probability the
        radioactive cloud spreads from state $s_r$ to $s'_r$.
        It captures the uncertainty of the  radioactive
        levels in the environment due to  noisy sensor
        readings and the variations in wind speed and direction.
    \item $P_{p_i}(s'_{p_i}|s, a_i)$ is the probability
        responder $p_i$ transitions to a new state $s'_{p_i}$
        when executing action $a_i$. For example, when a
        responder is asked to go to a new location, he or she
        may not end up there because he or she becomes tired, gets
        injured, or receives radiation doses that are life
        threatening.
    \item $P_{t_j}(s'_{t_j}|s, \vec{a})$ is the probability
        task $t_j$ being completed. A task $t_j$ can only be completed by a
        team of responders with the required types ($\Theta_{t_j}$) located at the
        same position as $t_j$.
\end{itemize}

Now,  if  $t_j$ is completed (i.e., in $s_{t_j}$, the status
$st_j$ is marked as ``dropped off'' and its position $(x_j, y_j)$
is within a safe zone), the team will be rewarded using
function $R$. The team is penalised if a responder
$p_i$ gets injured or receives a high dose of radiation (i.e., in
$s_{p_i}$, the health level $h_i$ is 0). Moreover, we attribute a
cost to each of the responders' actions since the each  action require them to
exert some effort (e.g., running or carrying objects).


Give the above definitions, a policy for the MMDP is a mapping from
states to joint actions, $\pi: S \rightarrow \vec{A}$ so that the
responders know which actions to take given the current state of
the problem. The quality of a policy $\pi$ is usually measured by
its expected value $V^\pi$, which can be computed recursively by
the Bellman equation:
\begin{equation}
  V^\pi(s^t) = R(s^t, \pi(s^t)) + \!\!\!\sum_{s^{t+1}\in S}\!\!\!
  P(s^{t+1}|s^t, \pi(s^t)) V^\pi(s^{t+1})
\end{equation}
where $\pi(s^t)$ is a joint action given $s^t$. The goal of solving
the MMDP is to find an optimal policy $\pi^*$ that maximises the
expected value with the initial state $s^0$, $\pi^* =
\arg\max_{\pi} V^\pi(s^0)$.

At each decision step, we assume the planning agent can fully
observe the state of the environment $s$ by collecting sensor
readings of the radioactive cloud and GPS locations of the
responders. Given a policy $\pi$ of the MMDP, a joint action
$\vec{a}=\pi(s)$ can be selected and broadcast to the responders
(as mentioned earlier). By so doing, each responder can be
instructed by the agent and know how to act in the field.
